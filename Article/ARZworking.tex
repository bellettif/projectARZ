\documentclass[preprint]{elsarticle}
\usepackage[margin=1in]{geometry}
\usepackage{hyperref}
\usepackage{amsmath}
\usepackage{mathrsfs}
\usepackage{float}
\graphicspath{{figures/}}
%\setlength\parindent{0pt}

\DeclareMathOperator{\card}{card}
\DeclareMathOperator{\trc}{trace}
\DeclareMathOperator{\mean}{mean}
\DeclareMathOperator{\bin}{bin}
\DeclareMathOperator{\lns}{lanes}
\DeclareMathOperator{\cnt}{count}

\makeatletter
\def\ps@pprintTitle{%
 \let\@oddhead\@empty
 \let\@evenhead\@empty
 \def\@oddfoot{\centerline{\thepage}}%
 \let\@evenfoot\@oddfoot}
\makeatother


%\usepackage{lineno,hyperref}
%\modulolinenumbers[5]

%\journal{Journal of \LaTeX\ Templates}

%%%%%%%%%%%%%%%%%%%%%%%
%% Elsevier bibliography styles
%%%%%%%%%%%%%%%%%%%%%%%
%% To change the style, put a % in front of the second line of the current style and
%% remove the % from the second line of the style you would like to use.
%%%%%%%%%%%%%%%%%%%%%%%

%% Numbered
%\bibliographystyle{model1-num-names}

%% Numbered without titles
%\bibliographystyle{model1a-num-names}

%% Harvard
%\bibliographystyle{model2-names.bst}\biboptions{authoryear}

%% Vancouver numbered
%\usepackage{numcompress}\bibliographystyle{model3-num-names}

%% Vancouver name/year
%\usepackage{numcompress}\bibliographystyle{model4-names}\biboptions{authoryear}

%% APA style
%\bibliographystyle{model5-names}\biboptions{authoryear}

%% AMA style
%\usepackage{numcompress}\bibliographystyle{model6-num-names}

%% `Elsevier LaTeX' style
\bibliographystyle{elsarticle-num}
%%%%%%%%%%%%%%%%%%%%%%%

\begin{document}

\begin{frontmatter}
\title{Prediction of traffic oscillation instability with spectral analysis of the Aw--Rascle--Zhang model}

%% Group authors per affiliation:
\author[add]{Authors}
\address[add]{address}
%\author{Elsevier\fnref{myfootnote}}
%\address{Radarweg 29, Amsterdam}
%\fntext[myfootnote]{Since 1880.}
%
%%% or include affiliations in footnotes:
%\author[mymainaddress,mysecondaryaddress]{Elsevier Inc}
%\ead[url]{www.elsevier.com}
%
%\author[mysecondaryaddress]{Global Customer Service\corref{mycorrespondingauthor}}
%\cortext[mycorrespondingauthor]{Corresponding author}
%\ead{support@elsevier.com}
%
%\address[mymainaddress]{1600 John F Kennedy Boulevard, Philadelphia}
%\address[mysecondaryaddress]{360 Park Avenue South, New York}

\begin{abstract}
This article will focus on macroscopic second-order models as opposed
to microscopic and mesoscopic frameworks.
\end{abstract}

\begin{keyword}

\end{keyword}

\end{frontmatter}

%\linenumbers

\section{Introduction}

As personal vehicle ownership increases globally, traffic congestion
continues to be a persistent problem. One goal of research in the physics of traffic is to gain a greater understanding of the phenomena, of which proper control would lead to mitigation of congestion. Traffic control strategies such as
ramp metering and variable speed limits are in place today, but improved models of traffic dynamics could lay the foundations for more efficient coordination of
control strategies. 

What constitutes an ``accurate" model often depends on the practitioners' requirements. A model appropriate for a given application must describe key phenomena in the dynamics of the system under scrutiny so that they can be controlled efficiently. For traffic, a commonly used class of models relies on hydrodynamic theory, modeling vehicular flow in an aggregate manner. However, control schemes are only as reliable as the models on which they are based. Given that each model can only reproduce some features of traffic flow, one challenge of traffic engineering is the selection of the model most appropriate for the application of interest. We begin with a short summary of the history of models leading to the model of interest in this article.

\subsection{First-order models}
The 1950's saw the development of the \textit{Lighthill--Whitham--Richards}
(LWR) model \cite{LW,Richards}, which became the seminal model for aggregate
traffic flow. The dynamics of this first-order scalar hyperbolic conservation law is governed by
\begin{equation}
\rho_t+\left(Q\left(\rho\right)\right)_x=0,
\end{equation}
where $\rho$ is the lineic vehicular density (veh/m), and $Q(\rho)$ is the empirically measured traffic flux (veh/s), also referred to as the fundamental diagram. Numerous phonomenological models exist for $Q(\rho)$,  such as the Greenshields model \cite{Greenshieldsflux}, the Greenberg model \cite{Greenbergflux}, the triangular model, \cite{Newell1993, daganzo1994cell}, power law models, \cite{Wang2005141, Papageorgiou198929, Flynn09self-sustainednonlinear} and many others. The simplicity of the LWR model enabled the formulation of numerical discretization
schemes such as the Godunov scheme \cite{godunov,Osher}, later renamed the cell transmission
model \cite{daganzo1994cell,daganzo1995cell} in the transportation community. 

However, first-order models have inherent shortcomings, most of which are discussed at length in \cite{Dag_requiem}, such as the failure to capture accurately shock structure, light traffic dynamics, and stop-and-go behavior, otherwise known as traffic oscillations. The last phenomenon in particular has attracted increasing attention in transportation research. For example jamitons, traffic jams that appear without the presence of a bottleneck, have been reproduced in contained experiments \cite{Jamitons2008,Flynn08onjamitons} and explained theoretically
as the result of both particular configurations of the traffic system
\cite{Flynn09self-sustainednonlinear} and
fuzzy fundamental diagrams \cite{Jamitons-multi-valued-fund}. 

%Empirical studies have focused on detecting and quantifying this effect on actual freeways \cite{zielke2008empirical}. Car-following behaviors and lane-changing are often cited as the cause of oscillations \cite{Mauch02freewaytraffic, lanechangin2005,lanechanging2007}. Several approaches have been developed in order to model these empirical facts.

\subsection{First-generation second-order models}
In an effort to refine traffic modeling, Payne and Whitham followed the same approach as in fluid dynamics, developing a higher-order model to capture momentum-related features \cite{payne1971models,whitham1974linear}. The \textit{Payne--Whitham} (PW) model consists of a mass conservation equation as in the LWR model,  
\begin{equation}
\rho_t + (\rho v)_x = 0, 
\end{equation}
where $\rho$ and $v$ are the density and velocity, respectively. This is seconded by a momentum equation,
\begin{equation}
v_t+vv_x+\frac{p'\left(\rho\right)}{\rho}\rho_x =\frac{V(\rho)-v}{\tau}+\nu v_{xx} ,
\end{equation}
where the pressure function, $p(\rho)$, is strictly increasing and the equilibrium speed, $V(\rho)$, is nonincreasing. Both are empirically measured and $V(\rho)$ satisfies $Q(\rho) = \rho V(\rho)$. The parameters $\tau$ and $\nu$ are nonnegative, representing the relaxation time and coefficient of viscosity, respectively. This approach, however, has issues in both the derivation of its equations, and its predictions \cite{Dag_requiem}. The PW model relies on the assumption that spacing and speed vary slowly, yielding negligible second and third derivatives for these quantities. This is contradictory with the observations of Newell \cite{Newell} given that the car-following model predicts sharp changes in these quantities. Mmoreover, the PW model violates the anisoptropy of traffic flow and can predict negative speeds \cite{Z}.

%Despite the common practice, modeling traffic as a gas is highly unrealistic as fluids feature isotropic propagation of information, while in traffic cars react mainly to downstream conditions. This is incompatible with the PW model since the slope of one of the characteristic curves is greater than cars' speeds. Also, at the upstream end of a jam with low density, traffic would flow in reverse if $\nu\neq0$, which is unrealistic. Finally, gas particles are physically identical whereas drivers have different habits and desired speeds. 

\subsection{Second-generation second-order models}
Despite the criticism expressed in \cite{Dag_requiem} regarding higher-order models inspired by fluid dynamics, Zhang proposed improved models to resolve issues plaguing earlier second-order models. In \cite{Zhang1998}, he proposed a modification of the momentum equation of the PW model to handle the issue of backward-propagating traffic.
%\begin{equation}
%v_t+vv_x+\rho\left(V'\left(\rho\right)\right)^{2}\rho_x=\frac{V\left(\rho\right)-v}{\tau},
%\end{equation}
%where $\rho V'\left(\rho\right)$ represents the information
%propagation speed in traffic. 
Soon after, Aw and Rascle \cite{AR} presented a model with the following momentum equation:
\begin{equation} \label{ARZEq}
(v+p(\rho))_t+v(v+p(\rho))_x=\frac{V(\rho)-v}{\tau}.
\end{equation}
Including the pressure term, $p(\rho)$, in a convective derivative guarantees that no information travels faster than the speed of the cars. Aw and Rascle demonstrated in \cite{AR} that ``with a suitable choice of function $p$," the above class of models avoids inconsistencies of earlier second-order models. Zhang proposed in \cite{Z} the same model with $p(\rho) = -V(\rho)$. With this choice of $p(\rho)$, the model is referred to as the \textit{Aw--Rascle--Zhang} (ARZ) model, which is the model used in the present article.

This model has since been thoroughly studied. The work in
\cite{AR} concluded that a relaxation term accounting for traffic
equilibrium was needed so the speed of cars would be determined by
the fundamental diagram and not the initial data. Rascle later proved in \cite{R_improved} that the relaxed model converges towards the LWR model as $\tau \rightarrow 0$. The first ARZ model requires a fundamental diagram extended to negative and maximum speeds, and zero and maximum densities so as to guarantee
solutions to the Riemann problem and stability with low densities \cite{Lebacque2007}. In \cite{HybridLagrangian2007}, 
discretization enabled extension of the AR model to a network setup with junctions and traffic lights. With an extended fundamental diagram, any Riemann problem for the ARZ model can be solved, allowing for a Godunov descretization scheme and numerical comparison
in which the ARZ model fits real data better than the LWR model \cite{GodunovARZ}.

%The ARZ model now has a legacy that started with the extended ARZ model \cite{lebacque2007generic}. An extension was presented in \cite{berthelin2008model} as a pressure-less limit of the AR model where drivers will not significantly slow down if the congestion is not heavy. 

More recently a generalized model proposed in \cite{Fan} corrects the fact that, in the ARZ model, several maximum densities co-exist, seemingly contradictory to the intuition that this quantity is uniquely determined by the characteristics of the freeway. Although more realistic, such a model does not allow proper linearization about an equilibrium, a procedure core key to making second-order models more tractable.

\subsection{Models appropriate for control}
As mentioned in \cite{Fan}, for congested regimes, the LWR model tends to offer a slightly better fit than the ARZ model with respect to empirical measurements but is outperformed for low densities. Ideally we would use a model suitable for all regimes so as to establish generic control strategies for traffic. Because such a model does not exist, we use instead the ARZ model linearized around arbitrary nominal conditions. 

Laplace transform and spectral analysis are powerful tools for control problems, providing a simple yet holistic representation of a system. In this regard, it is important to have a model that predicts oscillatory phenomena. Specific classes of second-order models, including the ARZ model in realistically account for traffic oscillations in dense traffic \cite{Greenberg_congestionredux}. Behavioral models such as in \cite{Newell}, and more recently \cite{Chen2012}, depict in a detailed fashion the effects of car-following
and lane-changing on freeway dynamics, effects often cited as the cause of oscillations \cite{Mauch02freewaytraffic, lanechangin2005,lanechanging2007}. However, second-order macroscopic models are most suited for our method. 

Fourier and Laplace transforms are not the only spectral operators useful for studying traffic. Approaches such as \cite{Zheng2011} attempt to model spectral features of traffic using wavelets to account for turbulent phenomena on various time scales. The method presented here differs in that it is driven mostly by macroscopic physics rather than data analysis.

\subsection{Similarities with Saint-Venant equations}
Our control analysis of the ARZ equations is strongly
inspired by the pioneering analysis of Litrico and Fromion for the Saint-Venant equations, see in particular thee monograph \cite{litrico2009modeling}. The Saint-Venant equations are as follows:
\begin{align}
Ty_t+q_x&=0, \\
v_t + vv_x + gy_x&=g\left(S_{b}-S_{f}\right),
\end{align}
where $v$ is the fluid velocity in a canal, $q$ the flux, $T$ the top width, $y$ the water height, $S_f$ the friction slope, and $S_b$ the bed slope. Note the structural similarity to the ARZ equations. In the equilibrium regime, the term $S_{b}-S_{f}\left(x,t\right)$ is zero under uniform flow, making it analogous to the relaxation term $\frac{V(\rho)-v}{\tau}$, designed to be zero at the equilibrium. Linearizing this system around an equilibrium point enables the use of spectral methods to design efficient control strategies for canals \cite{Litricoapplic} thus allowing hydrodynamicists to implement control theory effectively. Approximations for the low frequency domain decompose the transfer matrix into a combination of delay and integration components, allowing for design of efficient PI and $H_\infty$ controllers while enabling a simpler theoretical analysis of hydraulic systems \cite{litrico2009modeling, Litricoapplic}.

\subsection{Approach and contributions}
The present article adapts the corresponding spectral framework of \cite{litrico2009modeling} to the case of the ARZ equations so as to achieve a two-fold objective:

\begin{itemize}
\item We aim to develop strategies that enforce ease of use of the ARZ model for stability analysis and control. Analytical solutions to these non-linear equations are difficult to derive but linearization facilitates design of efficient control schemes with multiple inputs and outputs. We pay particular attention to the formulation of boundary conditions, to guarantee the well-posedness of the problem.
\item We assess the quality of the model by comparing its output with actual data collected in the field (using the NGSIM dataset). 
\end{itemize}

The contributions of this article are as follows:

\begin{itemize}
\item \textit{Modeling}: We derive the characteristic form by linearization and diagonalization of the ARZ model. This form highlights important features of the model, leading to the definition of a counterpart to the Froude number in hydrodynamics \cite{litrico2009modeling}, which separates free-flow and congested regimes.  

\item \textit{Spectral analysis}: From the charateristic form we derive the spectral form. Time domain responses derived from the spectral transfer matrices show that the linearized system is unstable in the free-flow regime and accounts for nonlinear wave propagation, giving rise to jamitons. These waves occur for an entire set of values of velocity, density, and flux and lead the linearized system away from its equilibrium point in the free-flow regime. 

\item \textit{Numerical vaildation}: A numerical experiment using NGSIM data is conducted to verify that linearization does not destroy realistic properties of the ARZ model. Previous studies also using NGSIM data to assess predictions of second-order models focused on averaged errors and only displayed predictions at a couple of points along the freeway \cite{GodunovARZ,Fan}. Here, we present an entire map of the states and conduct model assessment in a holistic manner, providing a complete analysis of the strengths and weaknesses of the model to be used for control. Our estimation procedure, unlike \cite{Fan}, does not rely on any assumption about the typical vehicle length or the safety distance factor. Additionally, no discretization scheme is needed and no grid size condition needs to be fulfilled. This procedure proves that the linearized model successfully accounts for traffic oscillations and also provides simple and consistent methods to calibrate the relaxation time, $\tau$.
\end{itemize}

\subsection{Organization of the article}
The rest of this article is organized as follows. In Section 2 we present the characteristic form of the ARZ model in several state variables, leading to the derivation of the spectral form of the flux and velocity system in section 3. We focus on these states in particular as they are the most easily observed and controlled in traffic. Properties of the spectral form in the two flow regimes are also analyzed. Section 4 focuses on numerical analysis. We present estimation procedures for $(v,q,\rho)$ and the model parameters, comparing empirical estimates with numerical predictions of the linearized model. Finally, we use \textit{Fast Fourier Transform} (FFT) to turn the spectral model into a prediction tool.


\section{The ARZ model} \label{ARZSection}

%\begin{itemize}
%\item physical description of ARZ model + usual form
%\item $(v, q)$ form, well-posedness 
%\end{itemize}

We consider the ARZ model with relaxation term. The model is shown here: 
\begin{align} 
\rho_t + (\rho v)_x &= 0, \label{ARZ1} \\
(v - V(\rho))_t + v(v - V(\rho))_x &=\dfrac{V(\rho) - v}{\tau} \label{ARZ2},
\end{align}
where $\rho$ is the density, $v$ is the velocity, $\tau$ is the relaxation time, and $V(\rho) = Q(\rho)/\rho$ is the equilibrium velocity profile, where $Q(\rho)$ is the density-flow relation given by the fundamental diagram. Without the relaxation term cars never reach the maximum allowable speed \cite{R_improved}. Note that at the equilibrium velocity this term is zero. 

In vector form the ARZ model is
\begin{equation} \label{ARZrhov}
\begin{pmatrix}
\rho \\ v
\end{pmatrix}_t
+ \begin{pmatrix}
v & \rho \\
0 & v + \rho V' (\rho)
\end{pmatrix}
\begin{pmatrix}
\rho \\ v
\end{pmatrix}_x = 
\begin{pmatrix}
0 \\ \dfrac{V(\rho) - v}{\tau}
\end{pmatrix}.
\end{equation}

With the appropriate variable change, we can rewrite the model in the density-flow and velocity-flow forms, the latter of which is most useful to us for practical control purposes. Using the flow relation $q = \rho v$ and \eqref{ARZrhov}, the density-flow form is
\begin{equation} \label{ARZrhoq}
\begin{pmatrix}
\rho \\ q
\end{pmatrix}_t
+ \begin{pmatrix}
0 & 1 \\
-\frac{q}{\rho} \left(\frac{q}{\rho} + \rho V'(\rho) \right) & 2\frac{q}{\rho} + \rho V'(\rho) 
\end{pmatrix}
\begin{pmatrix}
\rho \\ q
\end{pmatrix}_x = 
\begin{pmatrix}
0 & 0 \\ 
\frac{V(\rho)}{\tau} & -\frac{1}{\tau}
\end{pmatrix} 
\begin{pmatrix}
\rho \\ q
\end{pmatrix}.
\end{equation}

In the same manner we arrive at the velocity-flow form,
\begin{equation} \label{ARZvq}
\begin{pmatrix}
v \\ q
\end{pmatrix}_t
+ \begin{pmatrix}
v + \frac{q}{v} V'\left(\frac{q}{v}\right) & 0 \\
\frac{q}{v} \left( v + \frac{q}{v} V'\left(\frac{q}{v}\right)\right) & v
\end{pmatrix}
\begin{pmatrix}
v \\ q
\end{pmatrix}_x = 
\dfrac{1}{\tau}
\begin{pmatrix}
V\left( \frac{q}{v}\right) - v \\
\frac{q}{v}V\left( \frac{q}{v}\right) - q
% \dfrac{V\left( \frac{q}{v}\right) - v}{\tau} \\ 
% \dfrac{\frac{q}{v}V\left( \frac{q}{v}\right) - q}{\tau}
\end{pmatrix}.
\end{equation}

Although the $\left(\rho, q\right)$ and $\left(v, q\right)$ forms are less common in transportation engineering, they bear interesting similarities to hydrodynamical systems. Moreover, they are promising for sensing problems as loop detectors typically sense $\left(\rho, q\right)$ while GPS measurement generally yield estimates for $\left(v, q\right)$.


\subsection{Linearization}
We are interested in small deviations, $(\tilde{\rho}(x,t), \tilde{v}(x,t))$, from a given nominal profile. Consider the nominal solution $(\rho^*(x),v^*(x))(V(\rho^*) = v^*)$ satisfying $v_t = \rho_t = 0$. Then \eqref{ARZrhov} becomes
\begin{align}
v^* \rho^*_x + v^*_x\rho^* = 0, \\
( v^* + \rho^* V'( \rho^*) )v^*_x = \dfrac{V(\rho^*) - v^*}{\tau} = 0.
\end{align}
Then we must have $v^*_x=\rho^*_x=0$, so the solution is uniform along the road. 

Linearizing the ARZ model \eqref{ARZrhov} around the nominal solution described above, we obtain
\begin{equation} \label{rhovlin}
\begin{pmatrix}
\tilde{\rho} \\ \tilde{v}
\end{pmatrix}_t
+ \begin{pmatrix}
v^* & \rho^* \\
0 & v^* + \rho^* V' ( \rho^*) 
\end{pmatrix}
\begin{pmatrix}
\tilde{\rho} \\ \tilde{v}
\end{pmatrix}_x = 
\begin{pmatrix}
0 & 0 \\
\dfrac{V' (\rho^*)}{\tau} & -\frac{1}{\tau}
\end{pmatrix}
\begin{pmatrix}
\tilde{\rho} \\ \tilde{v}
\end{pmatrix}.
\end{equation}

Similarly for the density-flow system \eqref{ARZrhoq}, we linearize around the equilibrium $(\rho^*, q^*)(\rho^*V(\rho^*) = q^*)$ with deviations $(\tilde{\rho}(x,t), \tilde{q}(x,t))$. The linearized system is as follows:  
\begin{equation} \label{rhoqlin}
\begin{pmatrix}
\tilde{\rho} \\ \tilde{q}
\end{pmatrix}_t
+ \begin{pmatrix}
0 & 1 \\
\alpha^* \beta^* & \alpha^* - \beta^* 
\end{pmatrix}
\begin{pmatrix}
\tilde{\rho} \\ \tilde{q}
\end{pmatrix}_x = 
\begin{pmatrix}
0 & 0 \\
\delta & \sigma
\end{pmatrix}
\begin{pmatrix}
\tilde{\rho} \\ \tilde{q}
\end{pmatrix},
\end{equation}
where $\alpha^* = \frac{q^*}{\rho^*}$, $\beta^* = -\frac{q^*}{\rho^*} - \rho^* V'(\rho^*)$, $\delta = \dfrac{V(\rho^*)+\rho^*V'( \rho^*)}{\tau}$, and $\sigma = -\dfrac{1}{\tau}$.

Finally, for the velocity-flow system, we have
\begin{equation} \label{vqlin}
\begin{pmatrix}
\tilde{v} \\ \tilde{q}
\end{pmatrix}_t
+ 
\underset{A}{
	\underbrace{
		\begin{pmatrix}
		v^* + \frac{q^*}{v^*} V'\left(\frac{q^*}{v^*}\right) & 0 \\
		\frac{q^*}{v^*} \left( v^* + \frac{q^*}{v^*} V'\left(\frac{q^*}					{v^*}\right)\right) & v^*
		\end{pmatrix}
	}
}
\begin{pmatrix}
\tilde{v} \\ \tilde{q}
\end{pmatrix}_x 
= 
\underset{B}{
	\underbrace{
		\begin{pmatrix}
		-\dfrac{(v^*)^2+q^*V'\left(\frac{q^*}{v^*}\right)}{(v^*)^2 \tau} & 				\dfrac{V'\left(\frac{q^*}{v^*}\right)}{v^* \tau} \\
		-\dfrac{q^*\left((v^*)^2 + q^*V'\left(\frac{q^*}{v^*}\right)\right)}{(v^*)^3 \tau}  & \dfrac{q^*V'\left(\frac{q^*}{v^*}\right)}{(v^*)^2 \tau}
		\end{pmatrix}
	}
}
\begin{pmatrix}
\tilde{v} \\ \tilde{q}
\end{pmatrix}.
\end{equation}

\subsection{Characteristic form}
We diagonalize the linearized equations to obtain a more useful form of the model, which will then be treated in the spectral domain. 
%We can rewrite \eqref{rhovlin} in diagonal form. The eigenvalues of $A$ are 
%\begin{align}
%\lambda_1 &= v^*, \\
%\lambda_2 &= v^* + \rho^* V'( \rho^*).
%\end{align}
We begin with the density-flow system. Standard algebraic manipulations of the equations in \eqref{rhovlin} lead to
\begin{equation}
\begin{pmatrix}
\zeta_1 \\ \zeta_2
\end{pmatrix}_t
+ \begin{pmatrix}
\lambda_1 & 0 \\
0 & \lambda_2 
\end{pmatrix}
\begin{pmatrix}
\zeta_1 \\ \zeta_2
\end{pmatrix}_x
= \begin{pmatrix}
-\frac{1}{\tau} & 0 \\
-\frac{1}{\tau} & 0
\end{pmatrix}
\begin{pmatrix}
\zeta_1 \\ \zeta_2
\end{pmatrix},
\end{equation}
where $\zeta_1 = \tilde{v} - V'( \rho^* )\tilde{\rho}$ and $\zeta_2 = \tilde{v}$ are the Riemann invariants of the $(\rho, v)$ system, and $\lambda_1 = v^*$ and $\lambda_2 = v^* + \rho^* V'( \rho^*)$ are the eigenvalues. Note that $V'( \rho^*) < 0$ so $\lambda_2 \leq \lambda_1 = v^*$. Therefore this is consistent with the physical dynamics of the system as no waves travel faster than the equilibrium vehicle speed. 

We proceed in the same manner as above to diagonalize the $(\rho,q)$ system \eqref{rhoqlin}. The diagonal form is
\begin{equation} \label{rhoqlindiag}
\begin{pmatrix}
\chi_1 \\ \chi_2
\end{pmatrix}_t 
+ \begin{pmatrix}
\lambda_1 & 0 \\
0 & \lambda_2
\end{pmatrix}
\begin{pmatrix}
\chi_1 \\ \chi_2
\end{pmatrix}_x
= \begin{pmatrix}
-\frac{1}{\tau} & 0 \\
-\frac{1}{\tau} & 0
\end{pmatrix}
\begin{pmatrix}
\chi_1 \\ \chi_2
\end{pmatrix},
\end{equation}
where $\chi_1 = -\lambda_2 \tilde{\rho} + \tilde{q}$ and $\chi_2 = -\lambda_1 \tilde{\rho} + \tilde{q}$ are the characteristic variables in the $(\rho,q)$ system and the eigenvalues $\lambda_1$ and $\lambda_2$ are the same as in the density-velocity system due to the relation $q^* = \rho^*v^*$. 

Diagonalization of the velocity-flow system is more involved. Letting $\xi(x,t) = (\tilde{v}, \tilde{q})^T$, we can rewrite \eqref{vqlin} as
\begin{equation} \label{vqlinxi}
\eta_t + A\eta_x = B\eta.
\end{equation}
The eigenvalues of $A$ are $\lambda_1 = v^*$ and $\lambda_2 = v^* + \frac{q^*}{v^*} V'\left(\frac{q^*}{v^*}\right)$, consistent with the previous systems. Then $A$ can be diagonalized as follows
\begin{align}
A &= XDX^{-1}, \\
X &= \begin{pmatrix}
0 & \lambda_2-\lambda_1 \\
1 & \rho^*\lambda_2
\end{pmatrix}, \\
D &= \begin{pmatrix}
\lambda_1 & 0 \\
0 & \lambda_2
\end{pmatrix},\\
X^{-1} &= \begin{pmatrix}
\dfrac{\rho^*\lambda_2}{\lambda_1 - \lambda_2} & 1 \\
-\dfrac{1}{\lambda_1 - \lambda_2} & 0
\end{pmatrix}.
\end{align}

Define $\gamma(x,t) := X\eta(x,t)$. Hence \eqref{vqlinxi} can be rewritten as
\begin{equation} 
\gamma_t + \begin{pmatrix}
\lambda_1 & 0 \\
0 & \lambda_2
\end{pmatrix} \gamma_x = \begin{pmatrix}
-\frac{1}{\tau} & 0 \\
-\frac{1}{q^* \tau} & 0
\end{pmatrix} \gamma
\end{equation}
where 
\begin{equation}
\gamma = \begin{pmatrix}
\frac{\rho^*\lambda_2}{\lambda_1 - \lambda_2}\tilde{v} + \tilde{q} \\ 
-\frac{1}{\lambda_1 - \lambda_2}\tilde{v} 
\end{pmatrix}. 
\end{equation}

Let $\xi = (\xi_1, \xi_2)^T = (\chi_1, -q^*\chi_2)^T$. Then we have
\begin{equation} \label{vqlindiag}
\begin{pmatrix}
\xi_1 \\ \xi_2
\end{pmatrix}_t + \begin{pmatrix}
\lambda_1 & 0 \\
0 & \lambda_2
\end{pmatrix} \begin{pmatrix}
\xi_1 \\ \xi_2
\end{pmatrix}_x = \begin{pmatrix}
-\frac{1}{\tau} & 0 \\
-\frac{1}{\tau} & 0
\end{pmatrix} \begin{pmatrix}
\xi_1 \\ \xi_2
\end{pmatrix},
\end{equation}
and
\begin{equation} \label{eq:Riemannzeta}
\begin{pmatrix}
\xi_1 \\ \xi_2
\end{pmatrix} = \begin{pmatrix}
\dfrac{\rho^*\lambda_2}{\lambda_1 - \lambda_2}\tilde{v} + \tilde{q} \\ 
\dfrac{q^*}{\lambda_1 - \lambda_2}\tilde{v} 
\end{pmatrix} = 
\begin{pmatrix}
\dfrac{\rho^*\lambda_2}{\lambda_1-\lambda_2} & 1\\
\dfrac{\rho^*\lambda_1}{\lambda_1-\lambda_2} & 0
\end{pmatrix} \begin{pmatrix}
\tilde{v} \\ \tilde{q} 
\end{pmatrix}
\end{equation}

\subsection{The Traffic Froude Number}
In fluid mechanics, the Froude number is a dimensionless number which delineates the boundary between flow regimes \cite{Sturm, litrico2009modeling}. Using the eigenvalues of the system in the characteristic form, we are able to define a useful counterpart to this number.
Since $V(\rho)$ is nonincreasing function, we have $V'(\rho^*) \leq 0$. Thus there are two flow regimes: one in which $\lambda_1 \lambda_2 < 0$ and one characteristic curve travels downstream, and one in which $\lambda_1 \lambda_2 > 0$ and both characteristic curves travel upstream. 

We define the \textit{Traffic Froude Number} (TFN) as
\begin{equation}
F = \left\lvert\dfrac{\rho^*V'( \rho^*)}{v^*}\right\rvert.
\end{equation} 
Then we have
\begin{equation*}
\begin{cases}
F > 1 &\Rightarrow |\rho^*V'(\rho^*)| > v^* \quad \Rightarrow \lambda_2  <0 \\
F < 1 &\Rightarrow |\rho^*V'(\rho^*)| < v^* \quad \Rightarrow \lambda_2 > 0
\end{cases}.
\end{equation*}
Note also that $\lambda_2 = v^* + \rho^* V'( \rho^*) = \dfrac{Q(\rho^*)}{\rho^*} + \dfrac{\rho^*Q'(\rho^*)-Q(\rho^*)}{\rho^*} = Q'(\rho^*)$. Hence the system is in free-flow when $F<1$ and congestion when $F>1$. In hydrodynamics these regimes are referred to as the subcritical and supercritical regimes, respectively \cite{litrico2009modeling}

For traffic, the interpretation of the different regimes is somewhat different. Free flow regime corresponds to these situations where drivers are not slowed down by heavy traffic and go as fast as the authorized speed. The congested regime arises when traffic is denser and, because too many cars are present on the same freeway section, drivers slow down and eventually form traffic jam.


\section{Spectral analysis of the linearized ARZ model}
We now consider the $(v,q)$ system for the frequency domain analysis for practical control purposes.
\subsection{State-transition matrix}
Taking the Laplace transform of the diagonalized form \eqref{vqlindiag} we obtain 
\begin{equation}
\dfrac{\partial \hat{\zeta} (x,s)}{\partial x} = \mathscr{A}(s)\hat{\zeta}(x,s) + \mathscr{B}\zeta(x,t=0^-),
\end{equation}
where $\mathscr{A}(s) = A^{-1}(B - sI)$ and $\mathscr{B} = -A^{-1}$. 
The general solution to this ordinary differential equation is 
\begin{equation}
\hat{\zeta}(x,s) = \Phi(x,s)\hat{\zeta}(0,s) +  \Phi(x,s) \int^x_0 \Phi(\nu,s)^{-1} \mathscr{B} \zeta(\nu,0^-)\text{d} \nu,
\end{equation}
where $\Phi(x,s) = e^{\mathscr{A}(s)x}$ is the state-transition matrix. Assuming zero initial conditions we have 
\begin{equation} \label{TFRiemann}
\hat{\zeta}(x,s) = \Phi(x,s)\hat{\zeta}(0,s).
\end{equation}

To compute the exponential we diagonalize the matrix as 
\begin{equation}
\mathscr{A}(s) = \mathscr{X}(s)\mathscr{D}(s)\mathscr{X}^{-1}(s)
\end{equation} 
where  
\begin{align}
\mathscr{X}(s) &= \begin{pmatrix}
0 & \dfrac{\lambda_2 - (\lambda_1 - \lambda_2) \tau s}{\lambda_1} \\
1 & 1
\end{pmatrix}, \\
\mathscr{D}(s) &= \begin{pmatrix}
-\frac{s}{\lambda_2} & 0 \\
 0 & -\frac{1+\tau s}{\tau \lambda_1}
\end{pmatrix}.
\end{align}
Hence
\begin{equation}
\Phi(x,s) =  \mathscr{X}^{-1}(s)e^{\mathscr{D}(s)x}\mathscr{X}(s) =
\begin{pmatrix}
\phi_{11}(x,s) & \phi_{12}(x,s) \\ 
\phi_{21}(x,s) & \phi_{22}(x,s)
\end{pmatrix},
\end{equation}
with
\begin{subequations} \label{TFv0q0tovxqx}
\begin{align}
\phi_{11} &= e^{-\frac{x}{\tau \lambda_1}}e^{-\frac{x}{\lambda_1}s}, \\ 
\phi_{12} &= 0, \\
\phi_{21} &= \dfrac{\lambda_1 \left( e^{-\frac{x}{\tau \lambda_1}}e^{-\frac{x}{\lambda_1}s} - e^{-\frac{x}{\lambda_2}s}\right)}{\lambda_2 - \tau (\lambda_1 - \lambda_2)s}, \\
\phi_{22} &= e^{-\frac{x}{\lambda_2}s}.
\end{align}
\end{subequations}
\subsection{Free-flow case ($F1$)}
Consider the system in the free-flow regime. 

With $\zeta_1 (0,t)$ and $\zeta_2 (0,t)$ as the inputs and $\zeta_1(L,t)$ and $\zeta_2(L,t)$ as the outputs, the distributed transfer matrix is exactly the state-transition matrix $\Phi(x,s)$.\\

Using \eqref{eq:Riemannzeta}, we can write

\begin{equation} \label{vqfreeflow}
\begin{pmatrix}
\widetilde{v}(x,s) \\ \widetilde{q}(x,s)
\end{pmatrix} = \underbrace{ \begin{pmatrix}
\dfrac{\rho^*\lambda_2}{\lambda_1-\lambda_2} & 1\\
\dfrac{\rho^*\lambda_1}{\lambda_1-\lambda_2} & 0
\end{pmatrix}^{-1} \Phi(x,s) 
\begin{pmatrix}
\dfrac{\rho^*\lambda_2}{\lambda_1-\lambda_2} & 1\\
\dfrac{\rho^*\lambda_1}{\lambda_1-\lambda_2} & 0
\end{pmatrix} }_\text{$\Psi (x,s)$} \begin{pmatrix}
\widetilde{v}(0,s) \\ \widetilde{q}(0,s)
\end{pmatrix}
\end{equation}
with
\begin{subequations}
\begin{align}
\psi_{11}(x,s) &=\left(e^{-\frac{x}{\lambda_{1}\tau}}e^{-\frac{x}{\lambda_{1}}s}-e^{-\frac{x}{\lambda_{2}}s}\right)\frac{\alpha}{s+\alpha}+e^{-\frac{x}{\lambda_{2}}s}, \\
\psi_{12}(x,s)&=-\frac{1}{\rho^* \tau}\left(e^{-\frac{x}{\lambda_{1}\tau}}e^{-\frac{x}{\lambda_{1}}s}-e^{-\frac{x}{\lambda_{2}}s}\right)\frac{1}{s+\alpha}, \\
\psi_{21}(x,s)&=-\rho^{*} \tau\left(e^{-\frac{x}{\lambda_{1}\tau}}e^{-\frac{x}{\lambda_{1}}s}-e^{-\frac{x}{\lambda_{2}}s}\right)\frac{\alpha s}{s+\alpha}, \\
\psi_{22}(x,s)&=-\left(e^{-\frac{x}{\lambda_{1}\tau}}e^{-\frac{x}{\lambda_{1}}s}-e^{-\frac{x}{\lambda_{2}}s}\right)\frac{\alpha}{s+\alpha}+e^{-\frac{x}{\lambda_{1}\tau}}e^{-\frac{x}{\lambda_{1}}s}.
\end{align}
\end{subequations}

where $\alpha = -\dfrac{\lambda_2}{\tau(\lambda_1 - \lambda_2)}$. 

\subsubsection{Bode plots}

We generate Bode plots using the following parameters taken from \cite{Hofleitner}: $q_{\text{max}}$ = 1300 veh/h, $\rho_{\text{max}}$ = 0.1 veh/m, and $L$ = 100 m: The Greenshields Hamiltonian, $Q( \rho) = 4 \frac{q_{\text{max}}}{\rho_{\text{max}}^2}\rho (\rho_{\text{max}} - \rho)$, is used to approximate the fundamental diagram. For inhomogenous second-order models, the relaxation time, $\tau$, falls in the range of about 14-60 seconds \cite{Fan}. A relaxation time of $\tau$ = 15 s is used for the following simulations. We simulate for $\rho^* = 0.01$ veh/m.

The Bode plots for the physical variables are displayed in Figures \ref{fig:Magn_phase_physx} and \ref{Magn_spatial_physx}. For the Riemann invariants only $\phi_{21}(x,s)$ and $\phi_{22}(x,s)$ are represented in Figure \ref{fig:Magn_phase_diag} and Figure \ref{Magn_spatial_diag} ($\phi_{11}(x,s)$ and $\phi_{12}(x,s)$ are only delay functions).

\begin{figure}[H]
\centering
\begin{tabular}{cc}
\includegraphics[trim = 0mm 60mm 0mm 60mm, width=8cm]{IOv_-3to-1}
& 
\includegraphics[trim = 0mm 60mm 0mm 60mm, width=8cm]{IOq_-3to-1}
\tabularnewline
\end{tabular}
\caption{
Magnitude and phase bode plots for $\psi_{11}(L,s)$ and $\psi_{12}(L,s)$ (left) and for $\psi_{21}(L,s)$ and $\psi_{22}(L,s)$ (right). (Physical variables).
\label{fig:Magn_phase_physx}
}
\end{figure}


\begin{figure}
\centering
\begin{tabular}{cc}
\includegraphics[trim = 0mm 60mm 0mm 60mm, width = 8cm]{distr11_-3to-1}
&
\includegraphics[trim = 0mm 60mm 0mm 60mm, width = 8cm]{distr12_-3to-1}
\tabularnewline
Spatial magnitude Bode plot for $\psi_{11}(x,s)$.
&
Spatial magnitude Bode plot for $\psi_{12}(x,s)$.
\tabularnewline
\includegraphics[trim = 0mm 60mm 0mm 60mm, width = 8cm]{distr21_-3to-1}
&
\includegraphics[trim = 0mm 60mm 0mm 60mm, width = 8cm]{distr22_-3to-1}
\tabularnewline
Spatial magnitude Bode plot for $\psi_{21}(x,s)$.
&
Spatial magnitude Bode plot for $\psi_{22}(x,s)$.
\tabularnewline
\end{tabular}
\caption{Spatial magnitude Bode plots for physical variables\label{Magn_spatial_physx}}
\end{figure}

\begin{figure}[H]
\centering
\includegraphics[trim = 0mm 60mm 0mm 60mm, width = 110mm]{diagIOfreeflow}
\caption{Magnitude and phase Bode plots for $\phi_{11}(L,s)$ and $\phi_{21}(L,s)$.\label{fig:Magn_phase_diag}}
\end{figure}

\begin{figure}
\centering
\begin{tabular}{cc}
\includegraphics[trim = 0mm 60mm 0mm 60mm, width = 8cm]{diagdistr11freeflow}
&
\includegraphics[trim = 0mm 60mm 0mm 60mm, width = 8cm]{diagdistr21freeflow}
\tabularnewline
Spatial magnitude Bode plot for $\phi_{21}(x,s)$.
&
Spatial magnitude Bode plot for $\phi_{22}(x,s)$.
\tabularnewline
\end{tabular}
\caption{Spatial magnitude Bode plots for Riemann invariants\label{Magn_spatial_diag}}
\end{figure}

\subsubsection{Step responses}
We analyze the behavior of the system given step inputs $v(0,t)=v_{\text{step}}H(t)$ and $q(0,t)=q_{\text{step}}H(t)$, where $H(\cdot)$ is the Heaviside function. The step responses can be explicitly computed from the spectral responses:

\begin{align} 
v(x,t) &= v_{\text{step}}\left[e^{-\frac{x}{\lambda_1\tau}}\left(1 - e^{-a\left(t-\frac{x}{\lambda_1}\right)}\right)H\left(t-\frac{x}{\lambda_1}\right) + e^{-a\left(t-\frac{x}{\lambda_2}\right)}H\left(t-\frac{x}{\lambda_2}\right)\right] \notag \\
&\quad+ \dfrac{q_{\text{step}}}{\rho^* \tau}\left[- e^{-\frac{x}{\lambda_1\tau}}\left(1 - e^{-a\left(t-\frac{x}{\lambda_1}\right)}\right)H\left(t-\frac{x}{\lambda_1}\right) + \left(1 - e^{-a\left(t-\frac{x}{\lambda_2}\right)}\right)H\left(t-\frac{x}{\lambda_2}\right) \right] \\
q(x,t) &= v_{\text{step}}\rho^*\tau a\left[e^{-\frac{x}{\lambda_1\tau}}e^{-a\left(t-\frac{x}{\lambda_1}\right)}H\left(t-\frac{x}{\lambda_1}\right) - e^{-a\left(t-\frac{x}{\lambda_2}\right)}H\left(t-\frac{x}{\lambda_2}\right) \right] \notag \\
&\quad+ q_{\text{step}}\left[ e^{-\frac{x}{\lambda_1\tau}}e^{-a\left(t-\frac{x}{\lambda_1}\right)}H\left(t-\frac{x}{\lambda_1}\right) + \left(1 - e^{-a\left(t-\frac{x}{\lambda_2}\right)}\right)H\left(t-\frac{x}{\lambda_2}\right)\right]
\end{align}


\subsection{Congested regime ($F>1$)}
We now consider the system in the congested regime.

Using \eqref{TFRiemann} we can write 
\begin{equation} \label{vqcongested}
\begin{pmatrix}
\hat{\xi_{1}}(x,s)\\
\hat{\xi_{2}}(x,s)
\end{pmatrix} = \underbrace{
\Phi(x,s) \begin{pmatrix}
1 & 0\\
-\frac{\phi_{21}\left(L,s\right)}{\phi_{22}\left(L,s\right)} & \frac{1}{\phi_{22}}
\end{pmatrix}}_\text{$\Gamma (x,s)$}
\begin{pmatrix}
\hat{\xi_{1}}\left(0,s\right)\\
\hat{\xi_{2}}\left(0,s\right)
\end{pmatrix}.
\end{equation}
with 
\begin{subequations}
\begin{align}
\gamma_{11}\left(x,s\right)&=e^{-\frac{x}{\lambda_{1}\tau}}e^{-\frac{sx}{\lambda_{1}}}, \\
\gamma_{12}\left(x,s\right)&=0, \\
\gamma_{21}\left(x,s\right)&=\alpha\frac{\lambda_{1}}{\lambda_{2}}\left(e^{-\frac{x}{\lambda_{1}\tau}}e^{-\frac{sx}{\lambda_{1}}}-e^{-\frac{L}{\lambda_{1}\tau}}e^{-\frac{s}{\lambda_{2}}\left(x-L\frac{\lambda_{1}-\lambda_{2}}{\lambda_{1}}\right)}\right)\frac{1}{s+\alpha}, \\
\gamma_{22}\left(x,s\right)&=e^{-\frac{s\left(x-L\right)}{\lambda_{2}}}.
\end{align}
\end{subequations}

Note that equation \eqref{vqcongested} corresponds to a closed form solution of our initial system, written in spectral form.

\section{Numerical validation}

In this section we demonstrate the ability of the linearized ARZ equations to model the various nonlinear dynamics around a nominal operation point, The spectral form of the linearized
model provides a well-established control theoretic framework for designing control strategies for the system. Prior to using such techniques, it is necessary to assess how accurate the
model is in its linearized form. This section compares the prediction of the model with actual flow
and velocity data gathered from the well-known NGSIM data set.


\subsection{Data source: NGSIM trajectories}

We use the NSGIM trajectory data set for a section of the US-101 highway. The set gathers trajectories of vehicles sampled with a 10 Hz frequency thanks to high precision cameras. The data is pre-processed so as to take only cars into account; 45 minutes are recorded on a 650-meter long section with five lanes. The lanes are taken into account when computing the lineic density of vehicles $\rho$.
A map of the time evolution of speed along the section is given in Figure \ref{fig:NGSIM-trajectories}.
Only a subset of the spatial domain is used due to the presence of ramps, which breaks the homogeneity of the freeway. The viable domain is 200 meters long.

\begin{figure}[H]
\centering
\includegraphics[width=12cm]{Numerics/US-101_all_traj_low_res_mod}
\protect\caption{NGSIM trajectories. Color represents the measured speed of each
car in m/s.}
\label{fig:NGSIM-trajectories}
\end{figure}



\subsection{Reconstructing $(v,q)$ maps from NGSIM trajectories}

The NGSIM data set does not directly provide the values $v(t,x)$
and $q(t,x)$ in the resolution domain $\left[0,T\right]\times\left[0,L\right]$. To obtain macroscopic quantities out of the microscopic measurements, we divide the space-time grid into cells $\left(\left[i\Delta t,\, \left(i+1\right)\Delta t\right]\times\left[j\Delta x, \, (j+1)\Delta x\right]\right)_{i\in\left\{ 1\ldots n_{t}\right\} ,j\in\left\{ 1\ldots n_{x}\right\} }$, where $n_t$ and $n_x$ are the number of cells in time and space, respectively. We denote each cell as $\bin_{i,j}$. This operation consists of gathering corresponding data points into cells, then estimating the quantities of interest in each cell. 

Within each cell, a specific number of traces, or footprints of a vehicle along its trajectory, are available, and $\rho$, $v$, and $q$ are assumed to be constant. We present several formulae to map a set of traces to speed, flow, and density over the space-time grid. 

\textbf{Binning formula for $v$}: Since the speed is assumed to be constant in each cell, a straightforward estimate for the speed is the empirical average. The estimator for $v$ in $\bin_{i,j}$ is

\begin{equation}
\widehat{v}_{i,j}=\mean_{\trc \in \bin_{i,j}}(v(\trc)).
\end{equation}

\textbf{Binning formula for $\rho$}: By definition, the density of $\bin_{i,j}$ is  
\begin{equation}
\rho_{i,j}=\frac{1}{n_{\lns}\Delta x\Delta t}\iint_{\left(t,x\right)\in [i\Delta t, \,(i+1)\Delta t] \times [j\Delta x,\,(j+1)\Delta t]}\rho(x,t) \text{d}x \text{d}t.
\end{equation}

The position of each vehicle is recorded every 0.1 second. For each cell we count the number of traces and normalize it by the sampling rate. The contribution of a given vehicle to the density of a cell is proportional to the number of traces it has left in the cell. If the speed is assumed to be locally constant, this contribution is proportional to the time this vehicle spends in the cell and is consistent with the conservation of the total number of vehicles across all cells. Then we have the density estimator
\begin{equation}
\widehat{\rho}_{i,j}=\frac{1}{n_{\lns} \Delta x \Delta t \: \text{sampling rate}}\card ( \{ \trc \mid \trc \in \bin \} ),
\end{equation}
where $\card (\cdot)$ gives the number of elements in a set, i.e., its cardinal. 

\textbf{Binning formula for $q$}: By definition, $q=\rho v$, so a logical first estimate for $q$ in $\bin_{i,j}$ is 
\begin{equation}
\widehat{q}_{i,j}=\widehat{v}_{i,j}\widehat{\rho}_{i,j}.
\end{equation}

We can also approximate the flux through $\bin_{i,j}$ with a simple counting method. If a vehicle crosses spatial coordinate $\left(j+1\right)\Delta x$ between times $i\Delta t$ and $\left(i+1\right)\Delta t$, then it leaves a trace in both $\bin_{i,j}$ and $\bin_{i,j+1}$. Counting these vehicles and normalizing by the duration $\Delta t$ gives the estimator
\begin{equation}
\widehat{q}_{i,j}^{\cnt}=\frac{1}{n_{\lns}\Delta t}\: \card\left(\left\{ \text{id} \left(\trc\right)\mid \trc \in \bin_{i,j}\right\} \cap\left\{ \text{id}\left( \trc \right)\mid \trc\in \bin_{i,j+1}\right\} \right),
\end{equation}
where id$(\cdot)$ gives the identification number of a vehicle.



\subsubsection{Choosing the number of bins}

As the estimation formulae above rely on averaging, having a comfortable
number of points in each bin provides more stable estimates. It is worth mentioning that usual central limit theorem based reasoning for convergence of such estimates is flawed as several samples may correspond to the same vehicle or interacting vehicles, violating the independence assumption of the theorem. Proving the convergence of the estimates above lies beyond the scope of this article. As a rule of thumb we choose a discretization that guarantees that most bins will host more than $100$ traces. This is achieved with a $80\times80$ grid where the $10^{\text{th}}$ percentile of the number of traces in a given bin is $170$. Such a grid also yields a $10^{\text{th}}$ percentile of $56$ distinct vehicles per bin. The histograms of number of traces and vehicle per cell are given in Figure \ref{fig:Grid control}.

\begin{figure}[H]
\centering
\begin{tabular}{cc}
\includegraphics[width=8cm]{\string"/Numerics/Discretisation grid/traces_80_80\string".png} & \includegraphics[width=8cm]{\string"/Numerics/Discretisation grid/ids_80_80\string".png}\tabularnewline
Histogram of number of traces per cell. & Histogram of number of distinct vehicles per cell. \tabularnewline
\end{tabular}
\caption{Experimental justification for a $80\times 80$ cell based discretization
grid for the NGSIM data.}
\label{fig:Grid control}
\end{figure}

While our goal here is not to present theoretical proofs of the convergence of the binned estimators for $\left(v,\rho,q\right)$, it is nonetheless possible to check that the procedure is coherent. Two estimators are provided for $q$ that use radically
different techniques: the first relies on the average measured speed and the number of traces in a bin, while the other relies on counting vehicles transiting from a cell to another. Figure \ref{fig:Sanity-check} show that the scatter plot of $\widehat{q}_{i,j}^{\cnt}$ plotted against $\widehat{q}_{i,j}$ coincides nicely with the line $y=x$, validating the overall binning and estimation procedure above.

\begin{figure}[H]
\centering
\includegraphics[width=8cm]{\string"/Numerics/Sanity check/80_80_q_q_count\string".png}
\protect\caption{Sanity check for the estimation procedure. $\widehat{q}_{i,j}^{\text{count}}$
is plotted against $\widehat{q}_{i,j}$ across the grid of bins.
\label{fig:Sanity-check}}
\end{figure}



\subsection{Estimated values for $\left(v,q\right)$} 

To check how well the linearized ARZ model fits an actual dataset, we chose a bounded domain and compare the theoretical solution given by the second-order model and the observed data. Again we focus on the variables $v$ and $q$. Using the estimation procedure above, we compute fundamental diagrams from which we estimate the eigenvalues $\lambda_{1}$ and $\lambda_{2}$.
To calibrate the relaxation time $\tau$, we analyze the errors of predicted values of $v$ and $q$ for various $\tau$. The resulting maps of both the predicted and observed values highlight phenomena that the linearized model can and cannot account for. 

\textbf{Maps.} The estimates $\widehat{v}_{i,j}$, $\widehat{\rho}_{i,j}$, $\widehat{q}_{i,j}$, and $\widehat{q}_{i,j}^{\text{count}}$ are plotted on the discretized grid in Figure \ref{fig:Estimated-values}. Note that $\widehat{q}$ and $\widehat{q}^{\text{count}}$ give extremely similar results, so we may use $\widehat{q}^{\text{count}}$ from this point on. Damped oscillations and smoothly decaying values along characteristic lines are the main characteristic the practical implementation of the model should feature.

\textbf{Fundamental diagrams.} From the estimated values we can easily compute the fundamental diagrams given in Figure \ref{fig:Empirical-fundamental-diagrams}. We use the fundamental diagrams to calibrate the model parameters. Though the dataset used is dense, it covers only a small region of time and space. Thus, its small size is a potential flaw in our model parameter calibration as it is certain that
our measurements are highly correlated. This seems
to be confirmed by the fact that the fundamental diagrams below correspond only to the congested regime. \textit{OPTIONAL: Most of the points are concentrated
about the same region. This is not enough to guarantee that the estimated
quantities are reliable. However, NGSIM is to this day one of the
most comprehensive data sets of vehicle behavior on a freeway. It
is therefore one of the best ways one has to validate that a traffic
model is realistic. The fact that most points lie in the same region
is also a sign that the linearization hypothesis is reasonable
in that context. Observed deviations from the equilibrium are indeed
generally small. (The equilibrium, i.e. the linearization point, is estimated
below).}

\begin{figure}[H]
\centering
\begin{tabular}{cc}
\includegraphics[width=8cm]{\string"/Numerics/Maps data/80_80_v_map\string".png} & \includegraphics[width=8cm]{\string"/Numerics/Maps data/80_80_rho_map\string".png}\tabularnewline
\includegraphics[width=8cm]{\string"/Numerics/Maps data/80_80_q_map\string".png} & \includegraphics[width=8cm]{\string"/Numerics/Maps data/80_80_q_count_map\string".png}\tabularnewline
\end{tabular}
\protect\caption{Estimated values for $\left(v,q,\rho\right)$. Top left: $\widehat{v}_{i,j}$.
Top right: $\widehat{\rho}_{i,j}$. Bottom left: $\widehat{q}_{i,j}$.
Bottom right: $\widehat{q}_{i,j}^{\text{count}}$ .\label{fig:Estimated-values}}
\end{figure}

\begin{figure}[H]
\centering
\begin{tabular}{ccc}
\includegraphics[width=5cm]{\string"/Numerics/Maps data/80_80_fundamental_diagram_rho_q_count\string".png} & \includegraphics[width=5cm]{\string"/Numerics/Maps data/80_80_fundamental_diagram_v_q_count\string".png} & \includegraphics[width=5cm]{\string"/Numerics/Maps data/80_80_fundamental_diagram_v_rho\string".png}\tabularnewline
\end{tabular}
\protect\caption{Empirical fundamental diagrams. Left: $\left(\widehat{\rho},\widehat{q}^{\text{count}}\right)$.
Middle: $\left(\widehat{v},\widehat{q}^{\text{count}}\right)$. Right: $\left(\widehat{\rho},\widehat{v}\right)$.
\label{fig:Empirical-fundamental-diagrams}}
\end{figure}


\textbf{Calibration of $\lambda_{1}$ and $\lambda_{2}$, linearization point.} In Section \ref{ARZSection}, we found that $\lambda_{1}$ is exactly $v^*$ and $\lambda_{2}$ is the slope of the fundamental diagram at $v^*$. Thus to calibrate the eigenvalues we must find the linearization point. We estimate the linearization point using the Ordinary Least Squares method. Note the dataset used corresponds only to the congested regime and the fundamental diagram is almost affine. The estimator, $\widehat{\lambda}_1=\widehat{v}^*$ is chosen as the empirical mean of $\widehat{v}_{i,j}$. To estimate $\lambda_{2}$, we fit a linear model $\widehat{q}^{\text{count}}=b_{1}\widehat{\rho}+b_{0}+\varepsilon$, where $\varepsilon$
represents the noise in the model that would ideally be centered,
homoschedastic, and uncorrelated but is not practically. Then $\widehat{\lambda}_{2}=\widehat{b}_{1}$ and we take $\widehat{q}^*$ as the empirical average of $\widehat{q}^{\cnt}$. The ratio of $\widehat{q}^*$ and $\widehat{v}^*$ gives the estimate $\widehat{\rho}^*$.
Provided each estimator is convergent, the continuity of the functional
$\left(x,y\right)\rightarrow\frac{x}{y}$ on its domain guarantees the convergence of $\widehat{\rho}^*$. The empirical results are presented in Figure \ref{fig:Calibration-of-eigen-values}. The determination coefficient is poor but can be improved by filtering out outliers and gathering more data. Future work should include improving the quality of the estimation. Significance tests for the coefficients of the linear model are not presented. The assumptions they rely on about
the linear dependency between $\widehat{q}$ and $\widehat{v}$ are clearly
not respected here as the noise is auto-correlated. Further work should also to turn this rather heuristic method for estimating parameters
into a fully justified statistical procedure. Note that the goal of the present article is to provide a new model and corresponding spectral analysis, which we want to illustrate with state of the art data. Thus, development of statistical methods to handle this data is out of the scope of the present investigation.

\begin{figure}[H]
\centering
\includegraphics[width=8cm]{\string"/Numerics/Calibration of eigen values/Fundamental diagram fitting n = 80\string".png}
\protect\caption{Calibration of $\lambda_{1}$ and $\lambda_{2}$. The circle denotes the linearization point. The affine model used to estimate $\lambda_{2}$ and the linearization point is also plotted. The estimates are: $\widehat{\lambda}_{1}=8.96$ m/s, $\widehat{\lambda}_{2}=-4.37$ m/s,$\widehat{\rho}^{*}=0.049$ veh/m, $\widehat{v}^{*}=8.96$ m/s, $\widehat{q}^{*}=0.44$ veh/s, with $r^{2}=0.48$}
\label{fig:Calibration-of-eigen-values}
\end{figure}

\begin{figure}[H]
\centering
\includegraphics[width=8cm]{\string"/Numerics/Sanity check/80_80_q_q_count\string".png}
\protect\caption{Sanity check for the estimation procedure. $\widehat{q}_{i,j}^{\text{count}}$
is plotted against $\widehat{q}_{i,j}$ across the grid of bins.
\label{fig:Sanity-check}}
\end{figure}

\subsection{Verification of the spectral form}
In this section we demonstrate the performance of the spectral form as a prediction tool using the time domain responses derived from the transfer functions (see \ref{sub:Generic-computations}) and FFT. Since we are working with a linearized system, we can decompose boundary conditions then add predicted values inside the domain $\left[0,T\right]\times\left[0,L\right]$. Fourier decomposition of boundary conditions is here extremely accurate as the median relative errors for the interpolation of the values of $\xi_{1}\left(x=0,\right)$ and $\xi_{2}\left(x=L,\right)$ are respectively $2\%$ and $3\%$.
%A real signal $\left\{ f\left(t\right)\mid t\in\left[0,T\right]\right\} $ on one of the boundaries $\left\{ \left(x=0,t\right)\mid t\in\left[0,T\right]\right\} $ or $\left\{ \left(x=L,t\right)\mid t\in\left[0,T\right]\right\} $ is transformed into a periodic signal by infinite duplication and then turned into a Fourier series $\left\{ t\rightarrow\mu+\sum_{k=1}^{n}\beta_{k}\cdot \left(k\cdot wt+\phi_{k}\right)H\mid t\in\left[0,T\right]\right\} $. This process is known to be convergent with an infinite sum for any square integrable function. It is practically extremely accurate in our case even though the FFT only relies on a finite number of Fourier coefficients. 


\textbf{Simulated maps.} Since the spectral form presents information in the diagonalized basis, we need a conversion before we can compare the simulated results to the values estimated from the dataset.  
To make a comparison in the diagonalized basis, we first compute the estimated deviations from the equilibrium $\widehat{\widetilde{v}}_{i,j}=\widehat{v}_{i,j}-\widehat{v}^{*}$ and $\widehat{\widetilde{q}}_{i,j}=\widehat{q}_{i,j}-\widehat{q}^{*}$. Then the estimates for $\xi_{1}$ and $\xi_{2}$ are given by $\widehat{\xi}_{1_{i,j}}=\frac{\widehat{\rho}^{*}\widehat{\lambda}_{2}}{\widehat{\lambda}_{1}-\widehat{\lambda}_{2}}\widehat{\widetilde{v}}_{i,j}+\widehat{\widetilde{q}}_{i,j}$ and
$\widehat{\xi}_{2_{i,j}}=\frac{\widehat{\rho}^{*}\widehat{\lambda}_{1}}{\widehat{\lambda}_{1}-\widehat{\lambda}_{2}}\widehat{\widetilde{v}}_{i,j}$. To compare the physical variables, we compute the velocity and flow predictions by inverting \eqref{eq:Riemannzeta}: $\widetilde{q}=\xi_{1}-\frac{\lambda_{1}}{\lambda_{2}}\xi_{2}$,
$\widetilde{v}=\frac{\lambda_{1}-\lambda_{2}}{\rho^{*}\lambda_{1}}\xi_{2}$.

Figure \ref{fig:Data-versus-predicted.} shows important qualitative properties of the model. As expected, the model generally predicts with very good accuracy the decay of all quantities along their characteristic lines, a realistic feature that cannot be paralleled by first-order models. The general quality of the fit is rather good with most of the error on $v$ and $q$ in a $20\%$ range of the data's amplitude between minimum and maximum values. Furthermore the linearized second-order model manages to capture oscillations observed on the boundary and account for their decay accurately. 

\textbf{Calibration of $\tau$\label{sub:Calibration-of-tau}} For each $\tau$ we compute the \textit{mean absolute error} (MAE), or the average difference in absolute value between simulated and predicted values for each discretization cell. Since the quantities $v$ and $q$ are not physically homogeneous, it is not sensible to aggregate the errors over these quantities. However, $\xi_{1}$
and $\xi_{2}$ are both expressed in veh/s. Summing their MAE gives a reliable uni-dimensional index of the quality of the fit with respect
to $\tau$. This quantity is computed for different values of $\tau$
ranging from 5 to 80 seconds. The value offering the best fit
is $\tau^{*}=39.18$ s.

\begin{figure}[H]
\centering
\begin{tabular}{c}
\includegraphics[width=15cm]{\string"/Numerics/Maps simulated/vq_map_n=80_best_tau\string".png}\tabularnewline
\includegraphics[width=15cm]{\string"/Numerics/Maps simulated/xi_map_n=80_best_tau\string".png}
\end{tabular}
\protect\caption{Data versus predicted. Top Figure: $\left(v,q\right)$ domain, top row is $v$, bottom row is $q$. Bottom: $\left(\xi_{1},\xi_{2}\right)$
domain, top row is $\xi_{1}$, bottom row is $\xi_{2}$. First column: data. Middle column: predictions. Third column: error (difference between prediction and data).\label{fig:Data-versus-predicted.}}
\end{figure}

\begin{figure}[H]
\centering
\begin{tabular}{cc}
\includegraphics[width=8cm]{\string"/Numerics/Calibration of Tau/q_v_error_80\string".png} & \includegraphics[width=8cm]{\string"/Numerics/Calibration of Tau/xi_1_xi_2_error_80\string".png}\tabularnewline
MAE over $q$ and $v$ & MAE over $\xi_{1}$ and $\xi_{2}$ and sum of both MAE.\tabularnewline
\end{tabular}
\protect\caption{Calibration of $\tau$, one minimizes the sum of MAE over $\xi_{1}$
and $\xi_{2}$.}
\end{figure}


\subsubsection{Findings and conclusion from the numerical experiments}
The numerical experiments above have validated the accuracy of the linearized model and highlighted several of its core properties.
\begin{itemize}
\item The TFN delineates two regimes, congested, for $F > 1$ and free-flowing for $F < 1$. Linearization makes fully sense practically for the latter as the system remains about the equilibrium point.
\item The numerical experiments above show that the linearized ARZ model does reproduce NGSIM data very accurately for a homogeneous segment of the US-101 freeway. Oscillations are accounted for as well as their damping delay.
\item Finding that traffic becomes unstable for free-flowing regime apparently contradicts the findings of \cite{Jamitons-multi-valued-fund}. Here we do not use the same equilibrium velocity function and therefore it is not surprising we find that linear instability occurs for low densities instead of medium to high densities.
\item Our prediction method with the linearized ARZ model presents several advantages over the Godunov scheme. As long as FFT reproduces accurately boundary conditions, there is no need to check that CFL-like conditions are respected. 
\item Fourier is a smooth sinusoidal decomposition basis that only features spectral accuracy. Wavelet decomposition has been used so as to predict traffic both with spectral and temporal accuracy in \cite{Zheng2011} potentially enabling more precise prediction. However, such a method does not provide any easy way to design control schemes for traffic.
\end{itemize}


\section{Conclusion}

As the full nonlinear ARZ equations have no closed form solutions, they are difficult to analyze. The linearized equations enable the use of spectral methods presented here, allowing for elegantly simple yet powerful analysis tools. These equations are diagonalized, and solved explicitly using a spectral representation (transfer function). Using this approximation, we are able to analyze them around a nominal flow and characterize the oscillatory behavior of the solution. The linearized model is able to capture important features of the flow which first order models cannot. 

With the linearized ARZ model, we were also able to define the Traffic Froude Number. This quantity is computed using the eigenvalues of the system and characterizes the flow regime of the road section under consideration.

%We found the eigenvalues of the linearized system to be constant, prompting the definition of the Traffic Froude Number, $F$, which indicates the flow regime of the road section under consideration. Note that the Froude number inequalities delineating the supercritical and subcritical regimes in hydronamics are reversed for our definition of the TFN. 

Considering the transfer function of the linearized system of equations delineates the conditions for stability of the approximation about the equilibrium. The time domain responses we derive show that the system is unstable when $\lambda_2 > 0$ and stable when $\lambda_2 < 0$. In the free-flow regime, $F < 1$, oscillations increase and the system leaves the linear regime, while in the congested regime, $F>1$, oscillations decrease. In the latter case, the system remains in the linear regime and oscillations on boundary conditions are damped with an exponential rate along the characteristic lines. Thus, the TFN is also an indicator of linear stability.

The behavior predicted in congested regime for traffic does not present shocks and Fourier analysis cannot account for turbulent behavior as well as Wavelet transforms. However, our spectral domain study paves the way to applying standard linear system control theory to traffic, with a linearized second model that empirically is more accurate than its first order counterparts. Future work will therefore focus on controller design based on the spectral framework presented here.

%The present analysis also complements the wavelet-based approach used in \cite{Zheng2011}. Fourier transform based analysis is a powerful tool in our numerical experiments that does not require any CFL-like condition for stability. The linearity of the system clearly predicts smoother behavior than what is found with wavelet decomposition, and as we have shown, the resulting macroscopic predictions are fairly accurate.


%Second order models, because of their inherent complexity, were hard to analyze so far. No closed form solution was available and the frequency domain responses could not be written explicitly. Although it only offers an approximate representation of the system, our linearizing about an equilibrium has enabled a powerful yet simple analytical foundation for analysis. In particular, we have shown how to characterize important properties of the system such as Riemann Invariants and the \textit{Traffic Froude Number} that separates two classes of behaviors solutions to the ARZ equation can follow.
%
%For the free flowing regime ($F>1$), we have shown that around the linearization point, traffic systematically features oscillations of exponentially increasing amplitude that drive the system away from its equilibrium. This conclusion is different from those of \cite{Flynn09self-sustainednonlinear, Jamitons-multi-valued-fund}. Indeed, no assumption is made about the structure of the fundamental diagram except for the fact that the equilibrium speed decreases with traffic density. The oscillations we find occur for densities below a threshold whereas Jamitons are predicted to appear when the traffic density lies above a critical point whose value depends on the shape of the fundamental diagram. In such a regime, oscillations systematically propagate downstream whereas Jamiton waves can propagate upstream.
%
%In congested regime ($F<1$), traffic oscillations are damped with an exponential rate equal to $\tau$. This guarantees the linearized solution remains about the equilibrium around which the approximate system is accurate. Although decaying slowly, the solution does not feature the rapid dissipation that makes first order models somewhat inaccurate. In our numerical experiments, the linearized second order ARZ model provably accounts for phenomena that CTM based models could not represent. Indeed, the latter predict fast dissipative convergence of traffic density towards a homogeneous quantity along the freeway section under scrutiny which does not correspond to the NGSIM measurements.
%
%The present analysis also complements the wavelet-based approach used in \cite{Zheng2011}. Fourier transform based analysis is arguably a powerful tool in our numerical experiments that does not need any CFL-like condition to guarantee accuracy of predictions. Transfer functions provide simple analytical expression that are easy to interpret in both free flowing and congested regimes. The linearity of the system clearly predicts smoother behavior than what is found with wavelet decomposition. The resulting macroscopic predictions are practically accurate as we have shown.
%
%To conclude, we find that a freeway section evolves as a classical linear system therefore exploding, oscillating or converging towards $0$ at an exponential rate. Linearization has enabled the translation of the ARZ model from cumbersome non-linear PDE derivations towards a simple set of equations readily usable for control. Designing the corresponding schemes to alleviate congestion will be the subject of our further work.


\newpage
\appendix
\section{Generic computations for time domain to Laplace domain transforms
and vice versa\label{sub:Generic-computations}}

The aim is to derive the time domain responses of generic input signals
such as $t\rightarrow H\left(t\right)$ and $t\rightarrow \cos\left(wt+\phi\right)H\left(t\right)$
when multiplied in the Laplace domain by $\frac{1}{s+\alpha}$. This
then enables the computation of any response that decomposes in a
Fourier transform.


\subsection{Step function input}

The time domain input function is $H\left(t\right)$. One computes
the inverse Laplace transform of $s\rightarrow\frac{1}{s\left(s+\alpha\right)}$
which is 
\[
t\rightarrow\frac{1}{\alpha}\left(1-e^{-\alpha t}\right)H\left(t\right)
\]



\subsection{Phased cosine input}

The time domain input function is $\cos\left(wt+\phi\right)H\left(t\right)$.
One computes the inverse Laplace transform of $s\rightarrow\frac{1}{s+\alpha}\left\{ \frac{s}{s^{2}+\omega^{2}}\cos\left(\phi\right)-\frac{w}{s^{2}+w^{2}}\sin\left(\phi\right)\right\} $
which can be directly achieved in the time domain. Indeed, the result
is given by the convolution product $t\rightarrow\left(e^{-\alpha\cdot}H\left(\cdot\right)\star \cos\left(\omega\cdot+\phi\right)H\left(\cdot\right)\right)\left(t\right)$,
that is to say 
\[
t\rightarrow\frac{-e^{-\alpha\cdot t}\left(\alpha\cdot \cos\left(\phi\right)+w\cdot \sin\left(\phi\right)\right)+\alpha\cdot \cos\left(wt+\phi\right)+w\cdot \sin\left(wt+\phi\right)}{\alpha^{2}+w^{2}}H\left(t\right)=\kappa_{\alpha,w,\phi}^{\text{cos}}\left(t\right)
\]



\subsection{Fourier sum input}

Let the input be $t\rightarrow\mu H\left(t\right)+\sum_{k=1}^{n}\beta_{k}\cdot \cos\left(k\cdot wt+\phi_{k}\right)H\left(t\right)$.
The time domain response is therefore 
\[
t\rightarrow\frac{\mu}{\alpha}\left(1-e^{-\alpha t}\right)H\left(t\right)+\sum_{k=1}^{n}\beta_{k}\cdot\kappa_{\alpha,w,\phi}^{\text{cos}}\left(t\right)
\]



\subsection{Fourier decomposition and time domain responses for $\lambda_{2}>0$}

Let $\alpha=-\frac{\lambda_{2}}{\tau\left(\lambda_{1}-\lambda_{2}\right)}<0$.

\[
\left(\begin{array}{c}
\widehat{\xi_{1}}\left(x,s\right)\\
\widehat{\xi_{2}}\left(x,s\right)
\end{array}\right)=\Phi\left(x,s\right)\left(\begin{array}{c}
\widehat{\xi_{1}}\left(0,s\right)\\
\widehat{\xi_{2}}\left(0,s\right)
\end{array}\right)
\]
\\
with 
\[
\Phi\left(x,s\right)=\left[\begin{array}{cc}
e^{-\frac{sx}{\lambda_{1}}}e^{-\frac{x}{\lambda_{1}\tau}} & 0\\
-\alpha\frac{\lambda_{1}}{\lambda_{2}}\left(e^{-\frac{sx}{\lambda_{1}}}e^{-\frac{x}{\lambda_{1}\tau}}-e^{-\frac{sx}{\lambda_{2}}}\right)\frac{1}{s+\alpha} & e^{-\frac{sx}{\lambda_{2}}}
\end{array}\right]
\]
\\
implies the following fundamental responses for the system.


\subsubsection{Fundamental responses in time domain:}
\begin{itemize}
\item $\left(\begin{array}{c}
\xi_{1}\left(0,t\right)\\
\xi_{2}\left(0,t\right)
\end{array}\right)=\left(\begin{array}{c}
H\left(t\right)\\
0
\end{array}\right)$:

\begin{itemize}
\item $\xi_{1}\left(x,t\right)=e^{-\frac{x}{\lambda_{1}\tau}}H\left(t-\frac{x}{\lambda_{1}}\right)$
\item $\xi_{2}\left(x,t\right)=-\frac{\lambda_{1}}{\lambda_{2}}\left(e^{-\frac{x}{\lambda_{1}\tau}}\left(1-e^{-\alpha\left(t-\frac{x}{\lambda_{1}}\right)}\right)H\left(t-\frac{x}{\lambda_{1}}\right)-\left(1-e^{-\alpha\left(t-\frac{x}{\lambda_{2}}\right)}\right)H\left(t-\frac{x}{\lambda_{2}}\right)\right)$
\end{itemize}
\item $\left(\begin{array}{c}
\xi_{1}\left(0,t\right)\\
\xi_{2}\left(0,t\right)
\end{array}\right)=\left(\begin{array}{c}
0\\
H\left(t\right)
\end{array}\right)$:

\begin{itemize}
\item $\xi_{1}\left(x,t\right)=0$
\item $\xi_{2}\left(x,t\right)=H\left(t-\frac{x}{\lambda_{2}}\right)$
\end{itemize}
\item $\left(\begin{array}{c}
\xi_{1}\left(0,t\right)\\
\xi_{2}\left(0,t\right)
\end{array}\right)=\left(\begin{array}{c}
\cos\left(\omega t+\phi\right)\\
0
\end{array}\right)$:

\begin{itemize}
\item $\xi_{1}\left(x,t\right)=e^{-\frac{x}{\lambda_{1}\tau}}\cos\left(\omega\left(t-\frac{x}{\lambda_{1}}\right)+\phi\right)H\left(t-\frac{x}{\lambda_{1}}\right)$
\item $\xi_{2}\left(x,t\right)=-\frac{\lambda_{1}\alpha}{\lambda_{2}}\left(e^{-\frac{x}{\lambda_{1}\tau}}\kappa_{\alpha,\omega,\phi}^{\text{cos}}\left(t-\frac{x}{\lambda_{1}}\right)-\kappa_{\alpha,\omega,\phi}^{\text{cos}}\left(t-\frac{x}{\lambda_{2}}\right)\right)$
\end{itemize}
\item $\left(\begin{array}{c}
\xi_{1}\left(0,t\right)\\
\xi_{2}\left(0,t\right)
\end{array}\right)=\left(\begin{array}{c}
0\\
\cos\left(\omega t+\phi\right)
\end{array}\right)$:

\begin{itemize}
\item $\xi_{1}\left(x,t\right)=0$
\item $\xi_{2}\left(x,t\right)=\cos\left(\omega\left(t-\frac{x}{\lambda_{2}}\right)+\phi\right)H\left(t-\frac{x}{\lambda_{2}}\right)$
\end{itemize}
\end{itemize}

\subsection{Fourier decomposition and time domain responses for $\lambda_{2}<0$}

This time, $\alpha=-\frac{\lambda_{2}}{\tau\left(\lambda_{1}-\lambda_{2}\right)}>0$.

\[
\left(\begin{array}{c}
\widehat{\xi_{1}}\left(x,s\right)\\
\widehat{\xi_{2}}\left(x,s\right)
\end{array}\right)=\Phi\left(x,s\right)\left(\begin{array}{c}
\widehat{\xi_{1}}\left(0,s\right)\\
\widehat{\xi_{2}}\left(L,s\right)
\end{array}\right)
\]
\\
with

\[
\varGamma\left(x,s\right)=\left(\begin{array}{cc}
e^{-\frac{sx}{\lambda_{1}}}e^{-\frac{x}{\lambda_{1}\tau}} & 0\\
\alpha\frac{\lambda_{1}}{\lambda_{2}}\left(e^{-\frac{x}{\lambda_{1}\tau}}e^{-\frac{sx}{\lambda_{1}}}-e^{-\frac{L}{\lambda_{1}\tau}}e^{-\frac{s}{\lambda_{2}}\left(x-L\frac{\lambda_{1}-\lambda_{2}}{\lambda_{1}}\right)}\right)\frac{1}{s+\alpha} & e^{-\frac{s\left(x-L\right)}{\lambda_{2}}}
\end{array}\right)
\]
\\
implies the following fundamental responses for the system.


\subsubsection{Fundamental responses in time domain}
\begin{itemize}
\item $\left(\begin{array}{c}
\xi_{1}\left(0,t\right)\\
\xi_{2}\left(L,t\right)
\end{array}\right)=\left(\begin{array}{c}
H\left(t\right)\\
0
\end{array}\right)$:

\begin{itemize}
\item $\xi_{1}\left(x,t\right)=e^{-\frac{x}{\lambda_{1}\tau}}H\left(t-\frac{x}{\lambda_{1}}\right)$
\item $\xi_{2}\left(x,t\right)=\frac{\lambda_{1}}{\lambda_{2}}\left(e^{-\frac{x}{\lambda_{1}\tau}}\left(1-e^{-\alpha\left(t-\frac{x}{\lambda_{1}}\right)}\right)H\left(t-\frac{x}{\lambda_{1}}\right)-e^{-\frac{L}{\lambda_{1}\tau}}\left(1-e^{-\alpha\left(t-\frac{x-L\frac{\lambda_{1}-\lambda_{2}}{\lambda_{1}}}{\lambda_{2}}\right)}\right)H\left(t-\frac{x-L\frac{\lambda_{1}-\lambda_{2}}{\lambda_{1}}}{\lambda_{2}}\right)\right)$
\end{itemize}
\item $\left(\begin{array}{c}
\xi_{1}\left(0,t\right)\\
\xi_{2}\left(L,t\right)
\end{array}\right)=\left(\begin{array}{c}
0\\
H\left(t\right)
\end{array}\right)$:

\begin{itemize}
\item $\xi_{1}\left(x,t\right)=0$
\item $\xi_{2}\left(x,t\right)=H\left(t-\frac{x-L}{\lambda_{2}}\right)$
\end{itemize}
\item $\left(\begin{array}{c}
\xi_{1}\left(0,t\right)\\
\xi_{2}\left(L,t\right)
\end{array}\right)=\left(\begin{array}{c}
\cos\left(\omega t+\phi\right)\\
0
\end{array}\right)$:

\begin{itemize}
\item $\xi_{1}\left(x,t\right)=e^{-\frac{x}{\lambda_{1}\tau}}\cos\left(\omega\left(t-\frac{x}{\lambda_{1}}\right)+\phi\right)H\left(t-\frac{x}{\lambda_{1}}\right)$
\item $\xi_{2}\left(x,t\right)=\frac{\lambda_{1}\alpha}{\lambda_{2}}\left(e^{-\frac{x}{\lambda_{1}\tau}}\kappa_{\alpha,\omega,\phi}^{\text{cos}}\left(t-\frac{x}{\lambda_{1}}\right)-e^{-\frac{L}{\lambda_{1}\tau}}\kappa_{\alpha,\omega,\phi}^{\text{cos}}\left(t-\frac{x-L\frac{\lambda_{1}-\lambda_{2}}{\lambda_{1}}}{\lambda_{2}}\right)\right)$
\end{itemize}
\item $\left(\begin{array}{c}
\xi_{1}\left(0,t\right)\\
\xi_{2}\left(L,t\right)
\end{array}\right)=\left(\begin{array}{c}
0\\
\cos\left(\omega t+\phi\right)
\end{array}\right)$:

\begin{itemize}
\item $\xi_{1}\left(x,t\right)=0$
\item $\xi_{2}\left(x,t\right)=\cos\left(\omega\left(t-\frac{x-L}{\lambda_{2}}\right)+\phi\right)H\left(t-\frac{x-L}{\lambda_{2}}\right)$\end{itemize}
\end{itemize}

\section*{Acknowledgments}

%\section*{Appendices}
%\begin{itemize}
%\item Time domain solutions $(v, q)$
%\item Frequency domain for $(\rho, q)$, $(\rho, v)$
%\item Pre-processing of data
%\end{itemize}

\section*{References}

\bibliography{mybibfilecopy}

\end{document}