\documentclass[review]{elsarticle}
\usepackage{hyperref}
\usepackage{amsmath}
\setlength\parindent{0pt}
\makeatletter
\def\ps@pprintTitle{%
 \let\@oddhead\@empty
 \let\@evenhead\@empty
 \def\@oddfoot{\centerline{\thepage}}%
 \let\@evenfoot\@oddfoot}
\makeatother

%\usepackage{lineno,hyperref}
%\modulolinenumbers[5]

%\journal{Journal of \LaTeX\ Templates}

%%%%%%%%%%%%%%%%%%%%%%%
%% Elsevier bibliography styles
%%%%%%%%%%%%%%%%%%%%%%%
%% To change the style, put a % in front of the second line of the current style and
%% remove the % from the second line of the style you would like to use.
%%%%%%%%%%%%%%%%%%%%%%%

%% Numbered
%\bibliographystyle{model1-num-names}

%% Numbered without titles
%\bibliographystyle{model1a-num-names}

%% Harvard
%\bibliographystyle{model2-names.bst}\biboptions{authoryear}

%% Vancouver numbered
%\usepackage{numcompress}\bibliographystyle{model3-num-names}

%% Vancouver name/year
%\usepackage{numcompress}\bibliographystyle{model4-names}\biboptions{authoryear}

%% APA style
%\bibliographystyle{model5-names}\biboptions{authoryear}

%% AMA style
%\usepackage{numcompress}\bibliographystyle{model6-num-names}

%% `Elsevier LaTeX' style
\bibliographystyle{elsarticle-num}
%%%%%%%%%%%%%%%%%%%%%%%

\begin{document}

\begin{frontmatter}

\title{Title here}

%% Group authors per affiliation:
\author[add]{Authors}
\address[add]{address}
%\author{Elsevier\fnref{myfootnote}}
%\address{Radarweg 29, Amsterdam}
%\fntext[myfootnote]{Since 1880.}
%
%%% or include affiliations in footnotes:
%\author[mymainaddress,mysecondaryaddress]{Elsevier Inc}
%\ead[url]{www.elsevier.com}
%
%\author[mysecondaryaddress]{Global Customer Service\corref{mycorrespondingauthor}}
%\cortext[mycorrespondingauthor]{Corresponding author}
%\ead{support@elsevier.com}
%
%\address[mymainaddress]{1600 John F Kennedy Boulevard, Philadelphia}
%\address[mysecondaryaddress]{360 Park Avenue South, New York}

\begin{abstract}

\end{abstract}

\begin{keyword}

\end{keyword}

\end{frontmatter}

%\linenumbers

\section{Introduction}
\begin{itemize}
\item motivation (prediction, inference, control)
\item background (short lit review: LWR $\rightarrow$ ARZ progression, why second order models)
\item goals of paper
\item organization of paper
\end{itemize}
\section{The ARZ model}
%\begin{itemize}
%\item physical description of ARZ model + usual form
%\item $(v, q)$ form, well-posedness 
%\end{itemize}

The Aw-Rascle (AR) model \cite{AR} with relaxation term is
\begin{align} 
\rho_t + (\rho v)_x &= 0, \label{ARZ1} \\
(v + p(\rho))_t + v(v + p(\rho))_x &=\dfrac{V(\rho) - v}{\tau} \label{ARZ2},
\end{align}
where the ``pressure" function, $p(\rho)$, is a smooth, increasing function. This becomes the Aw-Rascle-Zhang (ARZ) model \cite{Z} with $p(\rho) = -V(\rho)$, where $V(\rho) = Q(\rho)/\rho$ is the equilibrium velocity profile, and $Q(\rho)$ is given by the fundamental diagram. \\

In vector form the ARZ model is

\begin{equation} \label{ARZrhov}
\begin{pmatrix}
\rho \\ v
\end{pmatrix}_t
+ \begin{pmatrix}
v & \rho \\
0 & v + \rho V' (\rho)
\end{pmatrix}
\begin{pmatrix}
\rho \\ v
\end{pmatrix}_x = 
\begin{pmatrix}
0 \\ \dfrac{V(\rho) - v}{\tau}
\end{pmatrix}.
\end{equation}

This system can be rewritten in the denisty-flow and velocity-flow forms, the latter of which is most useful to us for practical control purposes. Using $q = \rho v$ and \eqref{ARZrhov}, the density-flow form is

\begin{equation} \label{ARZrhoq}
\begin{pmatrix}
\rho \\ q
\end{pmatrix}_t
+ \begin{pmatrix}
0 & 1 \\
\frac{q}{\rho}  -\frac{q}{\rho} - \rho V'(\rho) & 2\frac{q}{\rho} + \rho V'(\rho)
\end{pmatrix}
\begin{pmatrix}
\rho \\ q
\end{pmatrix}_x = 
\begin{pmatrix}
0 & 0 \\ 
\frac{V(\rho)}{\tau} & -\frac{1}{\tau}
\end{pmatrix} 
\begin{pmatrix}
\rho \\ q
\end{pmatrix}.
\end{equation}

In the same manner we arrive at the velocity-flow form,

\begin{equation} \label{ARZvq}
\begin{pmatrix}
v \\ q
\end{pmatrix}_t
+ \begin{pmatrix}
v + \frac{q}{v} V'\left(\frac{q}{v}\right) & 0 \\
\frac{q}{v} \left( v + \frac{q}{v} V'\left(\frac{q}{v}\right)\right) & v
\end{pmatrix}
\begin{pmatrix}
v \\ q
\end{pmatrix}_x = 
\begin{pmatrix}
\dfrac{V\left( \frac{q}{v}\right) - v}{\tau} \\ 
\dfrac{\frac{q}{v}V\left( \frac{q}{v}\right) - q}{\tau}
\end{pmatrix}.
\end{equation}


\subsection{Linearization}
Consider the steady flow case $(\rho^*(x),v^*(x))(V(\rho^*) = v^*)$. Then we have
\begin{align}
v^* \frac{d\rho^*}{dx} + \frac{dv^*}{dx}\rho^* = 0, \\
\left( v^* + \rho^* V'( \rho^*) \right)\frac{dv^*}{dx} = \dfrac{V(\rho^*) - v^*}{\tau} = 0.
\end{align}
We must have $\frac{dv^*}{dx}=0$ else $v^* + \rho^* V'( \rho^*) = 0$. Then we have also $\frac{d\rho^*}{dx} = 0$. Hence the steady-state solution is uniform along the road. 

We are interested only in small deviations, $(\tilde{\rho}(x,t), \tilde{v}(x,t))$, from the equilibrium. We linearize the ARZ model \eqref{ARZrhov} about the steady-state described above. The linearized system is as follows 

\begin{equation} \label{rhovlin}
\begin{pmatrix}
\tilde{\rho} \\ \tilde{v}
\end{pmatrix}_t
+ \underbrace{\begin{pmatrix}
v^* & \rho^* \\
0 & v^* + \rho^* V' ( \rho^*) 
\end{pmatrix}}_\text{A}
\begin{pmatrix}
\tilde{\rho} \\ \tilde{v}
\end{pmatrix}_x = 
\begin{pmatrix}
0 & 0 \\
\dfrac{V' (\rho^*)}{\tau} & -\frac{1}{\tau}
\end{pmatrix}
\begin{pmatrix}
\tilde{\rho} \\ \tilde{v}
\end{pmatrix}
\end{equation}

We linearize \eqref{ARZrhoq} about the equilibrium $(\rho^*, q^*)(\rho^*V(\rho^*) = q^*)$ with deviations $(\tilde{\rho}(x,t), \tilde{q}(x,t))$. The linearized system is as follows 

\begin{equation} \label{rhoqlin}
\begin{pmatrix}
\tilde{\rho} \\ \tilde{q}
\end{pmatrix}_t
+ \begin{pmatrix}
0 & 1 \\
\alpha^* \beta^* & \alpha^* - \beta^* 
\end{pmatrix}
\begin{pmatrix}
\tilde{\rho} \\ \tilde{q}
\end{pmatrix}_x = 
\begin{pmatrix}
0 & 0 \\
\delta & \sigma
\end{pmatrix}
\begin{pmatrix}
\tilde{\rho} \\ \tilde{q}
\end{pmatrix},
\end{equation}

where $\alpha^* = \frac{q^*}{\rho^*}$, $\beta^* = -\frac{q^*}{\rho^*} - \rho^* V'(\rho^*)$, $\delta = \dfrac{V(\rho^*)+\rho^*V'( \rho^*)}{\tau}$, and $\sigma = -\dfrac{1}{\tau}$.

\begin{equation} \label{vqlin}
\begin{pmatrix}
\tilde{v} \\ \tilde{q}
\end{pmatrix}_t
+ \underbrace{\begin{pmatrix}
v^* + \frac{q^*}{v^*} V'\left(\frac{q^*}{v^*}\right) & 0 \\
\frac{q^*}{v^*} \left( v^* + \frac{q^*}{v^*} V'\left(\frac{q^*}{v^*}\right)\right) & v^*
\end{pmatrix}}_\text{A}
\begin{pmatrix}
\tilde{v} \\ \tilde{q}
\end{pmatrix}_x = 
\underbrace{\begin{pmatrix}
-\dfrac{(v^*)^2+q^*V'\left(\frac{q^*}{v^*}\right)}{(v^*)^2 \tau} & \dfrac{V'\left(\frac{q^*}{v^*}\right)}{v^* \tau} \\
-\dfrac{q^*\left((v^*)^2 + q^*V'\left(\frac{q^*}{v^*}\right)\right)}{(v^*)^3 \tau}  & \dfrac{q^*V'\left(\frac{q^*}{v^*}\right)}{(v^*)^2 \tau}
\end{pmatrix}}_\text{B}
\begin{pmatrix}
\tilde{v} \\ \tilde{q}
\end{pmatrix}.
\end{equation}

\subsection{Characteristic form}
We can rewrite \eqref{rhovlin} in diagonal form. The eigenvalues of $A$ are 
\begin{align}
\lambda_1 &= v^*, \\
\lambda_2 &= v^* + \rho^* V'( \rho^*).
\end{align}
Using the second equation in \eqref{rhovlin} and combining the equations of \eqref{rhovlin}, we have

\begin{equation}
\begin{pmatrix}
\zeta_1 \\ \zeta_2
\end{pmatrix}_t
+ \begin{pmatrix}
\lambda_1 & 0 \\
0 & \lambda_2 
\end{pmatrix}
\begin{pmatrix}
\zeta_1 \\ \zeta_2
\end{pmatrix}_x
= \begin{pmatrix}
-\frac{1}{\tau} & 0 \\
-\frac{1}{\tau} & 0
\end{pmatrix}
\begin{pmatrix}
\zeta_1 \\ \zeta_2
\end{pmatrix},
\end{equation}

where $\zeta_1 = \tilde{v} - V'( \rho^* )\tilde{\rho}$ and $\zeta_2 = \tilde{v}$. 

We proceed in the same manner as in the $(\rho, v)$ system to diagonalize \eqref{rhoqlin}. The diagonal form is

\begin{equation} \label{rhoqlindiag}
\begin{pmatrix}
\chi_1 \\ \chi_2
\end{pmatrix}_t 
+ \begin{pmatrix}
\lambda_1 & 0 \\
0 & \lambda_2
\end{pmatrix}
\begin{pmatrix}
\chi_1 \\ \chi_2
\end{pmatrix}_x
= \begin{pmatrix}
-\frac{1}{\tau} & 0 \\
-\frac{1}{\tau} & 0
\end{pmatrix}
\begin{pmatrix}
\chi_1 \\ \chi_2
\end{pmatrix},
\end{equation}

where 
\begin{align}
\chi_1 &= -\lambda_2 \tilde{\rho} + \tilde{q} \\
\chi_2 &= -\lambda_1 \tilde{\rho} + \tilde{q}
\end{align}

and the eigenvalues $\lambda_1$ and $\lambda_2$ are the same as in the density-velocity system.

Letting $\xi(x,t) = (\tilde{v}, \tilde{q})^T$, we can rewrite \eqref{vqlin} as
\begin{equation} \label{vqlinxi}
\xi_t + A\xi_x = B\xi.
\end{equation}

The eigenvalues of $A$ are the same as in the previous systems: 
\begin{align}
\lambda_1 &= v^*, \\
\lambda_2 &= v^* + \frac{q^*}{v^*} V'\left(\frac{q^*}{v^*}\right).
\end{align}
Then $A$ can be diagonalized as follows
\begin{align}
A &= XDX^{-1}, \\
X &= \begin{pmatrix}
0 & \lambda_2-\lambda_1 \\
1 & \rho^*\lambda_2
\end{pmatrix}, \\
D &= \begin{pmatrix}
\lambda_1 & 0 \\
0 & \lambda_2
\end{pmatrix},\\
X^{-1} &= \begin{pmatrix}
\dfrac{\rho^*\lambda_2}{\lambda_1 - \lambda_2} & 1 \\
-\dfrac{1}{\lambda_1 - \lambda_2} & 0
\end{pmatrix}.
\end{align}

Define $\chi(x,t) := X\xi(x,t)$. Hence \eqref{vqlinxi} can be rewritten as
\begin{equation} 
\chi_t + \begin{pmatrix}
\lambda_1 & 0 \\
0 & \lambda_2
\end{pmatrix} \chi_x = \begin{pmatrix}
-\frac{1}{\tau} & 0 \\
-\frac{1}{q^* \tau} & 0
\end{pmatrix} \chi
\end{equation}

where 
\begin{equation}
\chi = \begin{pmatrix}
\frac{\rho^*\lambda_2}{\lambda_1 - \lambda_2}\tilde{v} + \tilde{q} \\ 
-\frac{1}{\lambda_1 - \lambda_2}\tilde{v} 
\end{pmatrix}. 
\end{equation}

Let $\zeta = (\chi_1, -q^*\chi_2)^T$. Then we have
\begin{equation} \label{vqlindiag}
\zeta_t + \begin{pmatrix}
\lambda_1 & 0 \\
0 & \lambda_2
\end{pmatrix} \zeta_x = \begin{pmatrix}
-\frac{1}{\tau} & 0 \\
-\frac{1}{\tau} & 0
\end{pmatrix} \zeta,
\end{equation}
and
\begin{equation} \label{eq:Riemannzeta}
\zeta = \begin{pmatrix}
\dfrac{\rho^*\lambda_2}{\lambda_1 - \lambda_2}\tilde{v} + \tilde{q} \\ 
\dfrac{q^*}{\lambda_1 - \lambda_2}\tilde{v} 
\end{pmatrix} = 
\begin{pmatrix}
\dfrac{\rho^*\lambda_2}{\lambda_1-\lambda_2} & 1\\
\dfrac{\rho^*\lambda_1}{\lambda_1-\lambda_2} & 0
\end{pmatrix} \begin{pmatrix}
\tilde{v} \\ \tilde{q} 
\end{pmatrix}
\end{equation}

\subsection{Froude number}
In fluid mechanics the Froude number is a dimensionless number which delineates the boundary between flow regimes. Using the eigenvalues of the system in the characteristic form, we are able to define a useful analog to this number.
Since $V(\rho)$ is a decreasing function we have $V'(\rho^*) \leq 0$. Thus there are two flow regimes, where $\lambda_1 \lambda_2 < 0$ and one characteristic curve travels downstream, and where $\lambda_1 \lambda_2 > 0$ and both characteristic curves travel upstream. \\
Define $F := \dfrac{\rho^*V'( \rho^*)}{v^*}$. Then we have
\begin{equation*}
\begin{cases}
F > 1 &\Rightarrow |\rho^*V'(\rho^*)| > v^* \quad \Rightarrow \lambda_2  <0 \\
F < 1 &\Rightarrow |\rho^*V'(\rho^*)| < v^* \quad \Rightarrow \lambda_2 > 0
\end{cases}
\end{equation*}

Note also that $\lambda_2 = v^* + \rho^* V'( \rho^*) = \dfrac{Q(\rho^*)}{\rho^*} + \dfrac{\rho^*Q'(\rho^*)-Q(\rho^*)}{\rho^*} = Q'(\rho^*)$. Hence the system is in free flow when $F<1$ and congestion when $F>1$. 


\section{Frequency domain analysis}
\subsection{State-transition matrix}

\subsection{Free flow case}
[BC's, bode, step/sinusoid responses]

\subsection{Congested flow}
[BC's, bode, step/sinusoid responses]

\section{Numerical validation}
\begin{itemize}
\item Values for parameters $(\tau?)$
\item Convergence towards equilibrium of trajectories in $(v, q)$ domain
\item Spectral transform of data and response
\end{itemize}
\section{Results/discussion}


\section{Conclusion}
[further steps]

\section*{Acknowledgments}

\section*{Appendices}
\begin{itemize}
\item Time domain solutions $(v, q)$
\item Frequency domain for $(\rho, q)$, $(\rho, v)$
\item Pre-processing of data
\end{itemize}

\section*{References}

\bibliography{mybibfile}
\nocite*
\end{document}